%!TEX root = ../thesis.tex

\chapter{Einführung}
\label{chap:introduction}

Lorem ipsum dolor sit amet, consetetur sadipscing elitr, sed diam nonumy eirmod
tempor invidunt ut labore et dolore magna aliquyam erat, sed diam voluptua.
At vero eos et accusam et justo duo dolores et ea rebum. Stet clita kasd gubergren,
no sea takimata sanctus est Lorem ipsum dolor sit amet. Lorem ipsum dolor sit amet,
consetetur sadipscing elitr, sed diam nonumy eirmod tempor invidunt ut
labore et dolore magna aliquyam erat, sed diam voluptua. At vero eos et accusam et
justo duo dolores et ea rebum. Stet clita kasd gubergren, no sea takimata sanctus est Lorem ipsum dolor sit amet.


\section{Motivation}

Lorem ipsum dolor sit amet, consetetur sadipscing elitr, sed diam nonumy eirmod
tempor invidunt ut labore et dolore magna aliquyam erat, sed diam voluptua.
At vero eos et accusam et justo duo dolores et ea rebum. Stet clita kasd gubergren,
no sea takimata sanctus est Lorem ipsum dolor sit amet. Lorem ipsum dolor sit amet,
consetetur sadipscing elitr, sed diam nonumy eirmod tempor invidunt ut labore et dolore magna aliquyam erat, sed diam voluptua. At vero eos et accusam et justo duo dolores et ea rebum. Stet clita kasd gubergren, no sea takimata sanctus est Lorem ipsum dolor sit amet.

\section{Ziel der Arbeit}
Lorem ipsum dolor sit amet, consetetur sadipscing elitr, sed diam nonumy eirmod
tempor invidunt ut labore et dolore magna aliquyam erat, sed diam voluptua. At vero eos et accusam et justo duo dolores et ea rebum. Stet clita kasd gubergren, no sea takimata sanctus est Lorem ipsum dolor sit amet. Lorem ipsum dolor sit amet, consetetur sadipscing elitr, sed diam nonumy eirmod tempor invidunt ut labore et dolore magna aliquyam erat, sed diam voluptua. At vero eos et accusam et justo duo dolores et ea rebum. Stet clita kasd gubergren, no sea takimata sanctus est Lorem ipsum dolor sit amet.

\section{Aufbau der Arbeit}
Die Arbeit ist in drei Kapitel unterteilt. Das erste Kapitel beschäftigt sich primär mit den
Anforderungen an das System, welches den praktischen Teil dieser Arbeit darstellt.
Im zweiten Kapitel werden Grundlagen erläutert, die zum Verständnis der Technologieauswahl,
so wie für die Umsetzung des Systems ( Kapitel \ref{chap:frameworks}) erforderlich sind. Kapitel \ref{chap:tooling}
beschreibt Worflow Optimierungen und Deployment Strategien, speziell zur relevanten
Thematik der Cross Plattform Entwicklung.

Die Arbeit beschäftigt sich zunächst mit dem praktischen Teil der Arbeit.
Probleme Anforderungen
