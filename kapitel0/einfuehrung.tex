%!TEX root = ../thesis.tex

\chapter{Einführung}
\label{chap:introduction}

Bei der Planung von Softwareprojekten ist die Frage der Zielplattform meist eine der ersten und bedeutendsten Fragen,
die gelöst werden muss. Eine Entscheidung darüber, welche Geräte und Betriebssysteme eine Applikation erreichen soll,
beeinflusst nicht nur nachhaltig die Zielgruppe der Anwender, sie definiert darüber hinaus wichtige Grundvoraussetzungen der Entwicklung hinsichtlich Entwicklungsumgebung, Frameworks und Tools die überhaupt genutzt werden können.

Aufgrund der umfangreichen Hersteller- und Gerätevielfalt auf dem Markt sollte diese Entscheidung nicht leichtsinnig gefällt werden.
Eine Endnutzeranwendung kann für Webbrowser, Desktoprechner oder für mobile Geräte entwickelt werden.
Produkte von Softwaregiganten wie Spotify und Whatsapp bedienen beispielsweise alle diese Plattformen, um eine möglichst große Zielgruppe erreichen zu können.
So bieten beide Softwareanbieter jeweils Apps für gängige mobile Plattformen wie iOS, Android und Windowsphone
sowie Desktop Anwendungen für Mac und Windows und erreichen zusätzlich mithilfe ihrer Webanwendungen die Browser der Desktoprechner
und der der Mobilgeräte \cite{Spoti93:online} \cite{Whats74:online} \cite{Whats6:online}.

Eine native Entwicklung jeder Plattform erfordert großen Aufwand und resultiert damit in hohen
Entwicklungskosten, wodurch speziell kleinere Firmen aufgrund von strengen Budgetierungen oftmals
an Grenzen stoßen und
dazu gedrängt werden, ihre Zielgruppe einzuschränken oder einen plattformübergreifenden Ansatz für
die Entwicklung ihrer Anwendung zu verfolgen.

Im Rahmen dieser Arbeit werden verschiedene Technologien beleuchtet sowie eine
Cross Plattform Frontend Architektur für Webbrowser, Desktop- und Mobilgeräte in Form eines praktischen Teils
konzipiert und implementiert.

Ziel dieser Arbeit ist es, einen Eindruck der hybriden Anwendungsentwicklung mithilfe moderner
Webtechnologien zu vermitteln, indem Chancen und Risiken des Projekts, der Implementierung und relevanter Technologien aufgezeigt werden.

\vspace{0.6cm}

\noindent
Dazu ist die Arbeit in sieben Kapitel unterteilt. Das folgende Kapitel beschäftigt sich primär mit den
Anforderungen an das zu entwickelnde System.
Im dritten Kapitel werden Grundlagen erläutert, die zum Verständnis der Technologieauswahl
sowie für die Umsetzung des Systems, siehe Kapitel \ref{chap:frameworks}, erforderlich sind.
Kapitel \ref{chap:umsetzung} beschreibt die Konzeption und Implementierung der Frontend Architektur.
In einer abschließenden Reflexion, siehe Kapitel \ref{chap:umsetzung}, wird die Implementierung hinsichtlich den Anforderungen bewertet.
Außerdem wird besprochen, welches die ersten notwendigen technischen Schritte sind, um aus dem entwickelten Prototypen ein Produkt zu schaffen.
