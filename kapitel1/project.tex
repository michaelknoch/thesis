%!TEX root = ../thesis.tex

\chapter{Projekt \projectname}

In diesem Kapitel werden die Anforderungen an den praktischen Teil der Arbeit definiert.
\projectname wird im folgenden als Arbeitstitel für die Frontend Applikationen verwendet.

\section{Motivation}

Monitoring Tools dienen der Übewachung und Kontrolle bestehender Softwareprodukte.
Hierbei gibt es vielerlei Abläufe und Metriken, welche geprüft und ausgewertet werden können, um einem nutzer der Monitoring Applikation
eine aussagekräftige Übersicht über seine Systeme zu gewähren. Speziell nach dem Release eines Projektes oder nach dem Release neuer Patches
ist es meist unabdingbar das geupdatete System hinsichtlich Stabilität und Performance kritisch zu beobachten. Ein weiterer Aspekt, den es zu überwachen gilt,
ist die Skalierfähigkeit eines Produktes. Steigt die Aktivität einer Applikation, beispielsweise durch steigende Nutzerzahlen,
muss sichergestellt werden, dass steigende Antwort- und Ausfallzeiten, erkannt und richtig interpretiert werden, damit Skalierungsprobleme schnellstmöglich behoben werden können.
Handelt es sich bei dem zu überwachendem System um ein komplexes Zusammenspiel diverser Microservices, gestaltet sich die Herausforderung der manuellen Überwachung, beispielsweise anhand
Serverlogs, noch deutlich schwieriger.

Innerhalb einer Microservice Architektur ist es nicht notwendig, dass alle Services auf dem selben Server operieren.
Es ist möglich dass Services auf einer Vielzahl von Servern, womöglich an komplett unterschiedlichen geographischen Punkten miteinander agieren,
oder im Problemfall eben nicht miteinander agieren können. Instanzen einzelner Services können zur Laufzeit starten, stoppen oder womöglich abstürzen
und sollten dennoch den reibungslosen Ablauf der Applikation nicht behindern. Microservice Architekturen sind daher deutlich komplexer zu Überblicken,
daher ist ein Monitoring Tool, um die Stabilität und Performance einer solchen Architektur zu gewährleisten, meist stark von Nöten.


\section{Anforderungsanalyse}

Im nachfolgenden werden Anforderungen an das umzusetzende System gestellt um das in Motivation beschriebene Problem zu lösen.
