%!TEX root = ../thesis.tex

\chapter{Frameworks}

Im nachfolgenden werden die für den praktischen Teil der Arbeit genutzten Frameworks beschrieben.
Ziel dieses Kapitels ist es, die Auswahl der genannten Frameworks zu begründen und
ihren Funktionsumfang hinsichtlich der Anforderungen des Projekts zu untersuchen.
Des weiteren sollen mögliche Alternativen evaluiert werden.


\section{Angular2}
\subsection{Einführung}

Angular 2 ist die Nachfolgeversion des von Google entwickelten Javascript Framework Angular 1.
Die in 2009 veröffentlichte erste Version fand großen Anklang in der Community
und wurde als Basis für dynamische Single-page-Webanwendung verschiedenster Art und Größe genutzt.
In den seit Release vergangenen Jahren hat sich die Community um das Framework immer weiter vergrößert,
welche zur stetigen Weiterentwicklung und so zu dem Erfolg von Angular1 beigetragen hat.
Das Github Repository der ersten Version hat mittlerweile 1.489 Contributors mit nahezu 7000 gestellten Pull Requests. \cite{ng1-github}

Mit Angular 2 wurden einige Grundkonzepte überarbeitet um in eine komplett neue Richtung gehen zu können.
Ziel von Google ist es, ein komplett komponentenbasiertes leicht zu bedienendes Framework für moderne
Webanwendungen zu schaffen, welches bessere Performance und transparentere Interne Strukturen aufweisen soll, als die Vorgängerversion.
Eine Angular 2 Anwendung besteht daher aus einer Vielzahl diverser Komponenten, wodurch es möglich wird
Funktionalität zu kapseln, zu abstrahieren und wieder zu verwenden. Der Fokus hierbei liegt nicht nur auf Wiederverwendbarkeit innerhalb einer Codebasis.
Elemente der Anwendung sollen sowohl für den Browser, als auch für mobile Geräte, so wie für native Desktop Clients genutzt werden können.
Angular 2 soll im Vergleich zu seiner Vorgängerversion leichter zu lernen und nutzen sein,
so wie eine solide Basis auch für komplexere Webanwendungen bieten. \cite[11-12]{Angular2}

\subsection{Model View ViewModel}
\subsection{Komponenten}
\subsection{Dependency Injection}
\subsection{Rendering und Performance}


\section{Ionic}
\section{Electron}
