%!TEX root = ../thesis.tex

\chapter{Umsetzung}
\label{chap:umsetzung}

Im nachfolgenden wird die Konzeption und Umsetzung der in \ref{sec:anforderungsanalyse}
beschriebenen Anforderungen an das Projekt \projectname{} erläutert.
Der Schwerpunkt soll dabei speziell auf die Cross Plattform Architektur sowie auf die
Entwicklung der Komponenten hinsichtlich wiederverwendbarkeit gerichtet werden.


\section{Applikationsstruktur}

In diesem Kapitel wird die Konzeptphase der Plattformübergreifenden Applikationsstruktur
(Abbildung \ref{kapitel4/arch}) erläutert.
Um Endprodukte für Web, Desktop und App zu generieren werden die Frameworks Angular 2, Electron und das
Ionic \ac{SDK} verwendet. Die Codebase für die Web- und Desktop Anwendung lassen sich so weit vereinen,
dass diese innerhalb eines Repositories implementiert werden können.
Ein weiteres Repository beinhaltet die hybride Applikation,
welche mithilfe des Ionic 2 \ac{SDK} entwickelt wird. Obwohl Ionic 2 auf das Angular Framework aufbaut,
macht es Sinn den Web und App Code voneinander getrennt zu behandeln.
Gründe dafür sind zum einen konzeptionelle Unterschiede zwischen Web und App von \projectname{}.
Darüber hinaus unterscheiden sich die Navigation APIs der Frameworks, sowie die Menüführung der beiden Anwendungen.
Die \ac{UI} und Funktionalität wird daher innerhalb der Angular 2 (Web/Desktop)
Anwendung in Form von wiederverwendbaren Komponenten entwickelt und
Repository-übergreifend für die Verwendung im Ionic 2 Projekt verteilt.

\begin{itemize}
  \item Angular 2/Web und Desktop: github.com/michaelknoch/mia
  \item Ionic 2/App: github.com/michaelknoch/miamobile
\end{itemize}



\begin{figure}[h]
 \centering
 \includegraphics[width=\linewidth]{kapitel4/arch.png}
 \caption{Cross Plattform Architektur}
 \label{kapitel4/arch}
\end{figure}
\vspace{0.3cm}


\section{Verteilungsinfrastruktur}

Es wird eine Infrastruktur benötigt um Komponenten projektübergreifend verteilen zu können.
Möglich wäre die Konfiguration von Symlinks des lokalen Dateisystems um Komponenten in das Angular,
sowie Ionic Projekt zu integrieren. Symlinks lassen sich allerdings nicht versionieren, daher würde
für jeden neuen Entwickler ein komplexer initialer Konfigurationsaufwand entstehen.
Des Weiteren sollen Komponenten womöglich nicht nur in der Ionic App,
sondern in vielen weiteren auf Angular 2 basierenden Projekten wiederverwendet werden.
Eine Komponente oder ein Paket diverser Komponenten soll veröffentlicht und aktuell
gehalten werden und soll von Entwicklern genutzt werden können.

Es liegt nahe dieses Problem mithilfe eines Paketmanagers zu lösen. Angular und Ionic verwenden für das Management
ihrer Kern-Abhängigkeiten den \ac{NPM}.
Open Source Pakete können damit kostenfrei veröffentlicht und aktualisiert werden. Closed Source Pakete können
bereits für einen Aufpreis von 7\$ pro Monat genutzt werden.

\section{Komponentenentwicklung}

\subsection{Komponententypen}

Im folgenden werden Ansätze der Komponentenentwicklung bezüglich der in
\projectname{} erforderlichen Funktionalitäten entwickelt.


\subsubsection{generischer Ansatz}

Bei der Entwicklung generischer Komponenten ist der Begriff der Wiederverwendbarkeit von zentraler Bedeutung.
Komponenten werden im Vergleich zum spezifischen Ansatz nicht nur für einen Anwendungsfall konzipiert, sondern können
für diverse meist ähnliche Anwendungsbereiche genutzt werden. Dabei soll die Komponente nicht nur in der mobilen Ionic App,
sondern Projektübergreifend Verwendung finden.
Typisch für generische Angular 2 Komponenten ist die Verwendung des @Input Kanals.
Dabei werden zum einen Daten- und zur individualisierung nötige Konfigurationsobjekte in die Komponente eingeschleust.
Einige Komponenten der Applikation \projectname{} sind
generisch Entwickelt und OpenSource auf Github veröffentlicht.
Dadurch kann die Komponente von dritten genutzt und womöglich erweitert werden.


\subsubsection{spezifischer Ansatz}

Einige Komponenten lassen sich aufgrund ihrer spezifischen Funktionalität nicht generisch entwickeln.
Da sie nur einen spezifischen Anwendungsfall abdecken, können sie nur schwer in anderen Projekten wiederverwendet werden.
Dennoch werden diese innerhalb des Projekts \projectname{}
in der Angular 2 sowie in der Ionic 2 Anwendung verwendet.
Beispiele für spezifische Komponenten sind View- und Menükomponenten welche die Grundstruktur einer Applikation abbilden.

\subsection{MVC und Abhängigkeiten}

Im Idealfall können Komponenten voneinander gekapselt entwickelt werden,
sodass sie völlig eigenständig nutzbar sind und untereinander keine Abhängigkeiten aufweisen.
Jede Komponente beinhaltet individuelle \ac{MVC} Schichten für die Implementierung der Funktionalität von der View bis zur Datenschicht.
Modelklassen (Typisierung) und Services befinden sich innerhalb der Komponente und nicht innerhalb eines globalen Model Layers.
Allerdings entstehen im Aufbau einer Angular Applikation schnell komponentenübergreifende Abhängigkeiten.
Wird Funktionalität in Services, welcher von mehreren Komponenten genutzt wird, implementiert
so sollte dieser nicht in die Komponente gekapselt werden, sondern sich in einer sichtbaren Schicht zur globalen Verwendung befinden.
Ferner ist es warscheinlich, dass Typescript Modelklassen (Typisierung) ebenfalls in mehreren Komponenten verwendet werden.
Dennoch sollten diese nicht redundant implementiert,
sondern Zentral zur Verfügung gestellt werden. Es stellt sich also heraus,
dass nicht alle Komponenten der Applikation \projectname{} als eigenständige Module ausgeliefert werden können,
sondern es durchaus Sinn ergibt Komponentenpakete über die Infrastruktur zu verteilen.


\subsection{Komponenten Generator}

Um den Entwicklungsprozess der Anwendung zu beschleunigen wurde mithilfe von yeoman.io ein Komponenten-Generator entwickelt.
Dieser ermöglicht die Grundstruktur neuer Komponenten anhand diverser Optionen in kurzer Zeit auszuliefern.
Dabei wird der Name sowie der Auslieferungsort der neuen Komponente während der Generierung im Terminal erfragt.
Zusätzlich kann definiert werden, ob externe \ac{HTML} und \ac{CSS} Dateien angelegt, oder ob diese inline in der Komponente genutzt werden sollen.
Der Generator wurde ebenfalls Open Source entwickelt und auf Github veröffentlicht
(github.com/michaelknoch/generator-ng2-comp).

\subsection{Implementierung}

Im nachfolgenden werden die in \projectname{} entwickelten Komponenten in drei verschiedene Kategorien unterteilt.

\subsubsection{Views}



\subsubsection{Elemente}

\subsubsection{Bibliothek Interfaces}

In \projectname werden die externen Bibliotheken Chart.js und Cytoscape.js verwendet.
Für deren Nutzung sind Interfaces in Form von Komponenten implementiert.
Das bedeutet, dass Instanzen von Chart.js und Cytoscape.js nicht anhand des globalen \ac{DOM} \ref{component-interface-1},
sondern mittels Angular Komponenten realisiert werden.

\newpage
\paragraph{Chart.js}
ist eine Bibliothek für das Zeichnen komplexer Diagramme.
Es wird in \projectname{} hauptsächlich für die visualisierung der Daten für die Metrik Sektion verwendet.
Chart.js ist responsive implementiert, d.h. die Charts können sowohl für den Desktop
als auch für die Mobile Applikation verwendet werden \cite{Chart80:online}.

\vspace{0.3cm}
\lstinputlisting[language=JavaScript,label=component-interface-1,caption=Chart.js ohne Komponenten Interface]{kapitel4/chartjs-oldschool.js}

\lstinputlisting[language=JavaScript,label=component-interface-2,caption=Chart.js mit Komponenten Interface]{kapitel4/chartjs-new.js}
\vspace{0.3cm}
In \ref{component-interface-2} wird die Nutzung des Komponenten Interface \textbf{ng2-charts} (github.com/valor-software/ng2-charts) verdeutlicht.
Die Komponente wird von Valor Software entwickelt und steht quelloffen unter der MIT Lizenz zur Verfügung \cite{valor6:online}.

Die Komponente erweist sich als sehr hilfreich, da das Chart bereits durch die simple Verwendung des Komponenten-Selektors (base-chart) instanziiert werden kann.
Des Weiteren werden die zu visualisierenden Daten sowie ein optionales Konfigurationsobjekt über den Selektor an die Komponenten übergeben.
Ändert sich das Datenobjekt wird das Chart automatisch aktualisiert.


\newpage
\paragraph{Cytoscape.js}
ist eine Bibliothek zur Visualisierung und Analyse komplexer Netzwerke.
Diese findet in \projectname{} Verwendung in der Implementierung der Graphen-Sektion,
in welcher Interaktionen zwischen Services visualisiert werden.
Um eine mit ng2-charts vergleichbar bequeme Nutzung der Graphen-Bibliothek zu ermöglichen,
wurde eine eigene Interface-Komponente entwickelt und als Open Source auf Github veröffentlicht.

\lstinputlisting[language=JavaScript,label=ng2-cytoscape,caption=ng2-cytoscape Implementierung]{kapitel4/ng2-cytoscape.ts}

Listing \ref{ng2-cytoscape} beinhaltet die Implementierung von ng2-cytoscape.
Um Übersicht zu gewährleisten wird für die Implementierung nicht relevanter Inhalt einiger Konfigurationsobjekte mit (...) abgekürzt.



\section{Komponentenverteilung}

Um redundanten Code zu vermeiden sollen Komponenten aus der Angular 2 Applikation in der Ionic 2 App Verwendung finden.
Hierzu soll ein möglicher Workflow entwickelt und evaluiert werden, welcher Änderungen
der Angular 2 Codebase als Updates für das Ionic Projekt propagiert.

\subsection{Vorbereitung und Veröffentlichung}

Die zu verteilenden Komponenten sind in Typescript geschrieben,
daher müssen sie entweder im Zielprojekt in die Transpilierung mit einbezogen werden,
oder bereits vor der Verteilung transpiliert und als JavaScript Paket veröffentlicht werden.
Interessant hierbei sind die Optionen \textbf{sourceMap} und \textbf{declaration} des Typescript Transpilers.
Sind diese aktiviert, werden neben den transpilierten .js Javascript Dateien jeweils d.ts und .map Dateien abgelegt.

\paragraph{Declaration(d.ts)}

Eine d.ts Datei wird als TypeScript Declaration File bezeichnet.
Es beschreibt Implementierungen, welche in JavaScript geschrieben sind oder von TypeScript zu JavaScript transpiliert wurden.
Das Declaration File ermöglicht die Verwendung von JavaScript Code, beispielswiese einer externen Bibliothek,
in ein Projekt, welches in TypeScript geschrieben ist. Das Declaration File fungiert dabei als Interface
für die JavaScript Implementierung und gewährleistet statische Typisierung
und Autovervollständigung in unterstützenden IDEs.

Im TypeScript Transpilierungsprozess können Declaration Files mithilfe der Option \textbf{declaration} generiert werden.
Für viele populäre JavaScript Bibliotheken wurden bereits Declaration Files, von der Community oder von dem ursprünglichen Autor, nachgeliefert.
\cite[471]{EssentialTS}

\paragraph{SourceMap (.map)}

Durch die Verwendung von *-to-JavaScript Compiler/Transpiler und Minifizierungstools, ensteht ein Problem, welches SourceMaps zu lösen versuchen.
Der zur Entwicklungszeit geschriebene Code ist nicht der Selbe, welcher zur Laufzeit im Browser ausgeführt wird, da dieser transpiliert und womöglich minifiziert wurde.
Wenn nun Fehler der Applikation zur Laufzeit identifiziert werden, können diese nicht mehr auf den Ursprungscode abgebildet werden.
Der Typescript Compiler beinhaltet einen SourceMaps Generator, welcher beim Transpilevorgang .map Dateien erzeugt,
welche dabei als Referenztabelle zwischen Quell und Zielcode fungieren.
Öffnet man nun die Entwicklerkonsole in einem Browser, welcher SourceMaps unerstützt, kann man den ursprünglichen Code inspizieren.
\cite{Using97:online}



\subsection{Verwendung im Ionic Projekt}

\ac{NPM} Module werden mit dem Befehl \textbf{npm install} installiert.
Sie stehen im Projekt innerhalb der Node Modules zur Verfügung und können in das Projekt importiert und verwendet werden.




\paragraph{HTML und CSS Inlining}
Zudem werden \ac{HTML} und \ac{CSS} referenzen der Komponente in Inline-Strings konvertiert.
Damit werden Konflikte bezüglich relativen und absoluten Referenz-Pfaden verhindert,
da diese zentral in einer Angular Applikation konfiguriert werden.
Sind Style und Markup der Komponente als Inline-String definiert,
wird diese Konfiguration bezüglich der verteilten Komponenten hinfällig.
\cite{ludoh30:online}

\section{Auslieferung}
\subsection{Web}
\subsection{Dekstop}
\subsection{App}
