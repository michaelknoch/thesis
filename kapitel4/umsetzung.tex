%!TEX root = ../thesis.tex

\chapter{Umsetzung}
\label{chap:umsetzung}

\section{Konzeption}


\section{Komponenten verteilen und vewenden}

Um redundanten Code zu vermeiden sollen Komponenten aus der Angular 2 Applikation in der Ionic 2 App Verwendung finden.
Hierzu soll ein möglicher Workflow entwickelt und evaluiert werden, welcher Änderungen
der Angular 2 Codebase als Updates für das Ionic Projekt propagiert.

\subsubsection{Komponenten vorbereiten}

Die zu verteilenden Komponenten sind in Typescript geschrieben,
daher müssen sie entweder im Zielprojekt in die Transpilierung mit einbezogen werden,
oder bereits vor der Verteilung transpiliert und als JavaScript Paket veröffentlicht werden.
Interessant hierbei sind die Optionen \textbf{sourceMap} und \textbf{declaration} des Typescript Transpilers.
Sind diese aktiviert, werden neben den transpilierten .js Javascript Dateien jeweils d.ts und .map Dateien abgelegt.

\paragraph{Declaration(d.ts)}

Eine d.ts Datei wird als TypeScript Declaration File bezeichnet.
Es beschreibt Implementierungen, welche in JavaScript geschrieben sind oder von TypeScript zu JavaScript transpiliert wurden.
Das Declaration File ermöglicht die Verwendung von JavaScript Code, beispielswiese einer externen Bibliothek,
in ein Projekt, welches in TypeScript geschrieben ist. Das Declaration File fungiert dabei als Interface
für die JavaScript Implementierung und gewährleistet statische Typisierung
und Autovervollständigung in unterstützenden IDEs.

Im TypeScript Transpilierungsprozess können Declaration Files mithilfe der Option \textbf{declaration} generiert werden.
Für viele populäre JavaScript Bibliotheken wurden bereits Declaration Files, von der Community oder von dem ursprünglichen Autor, nachgeliefert.
\cite[471]{EssentialTS}

\paragraph{SourceMap (.map)}

Durch die Verwendung von *-to-JavaScript Compiler/Transpiler und Minifizierungstools, ensteht ein Problem, welches SourceMaps zu lösen versuchen.
Der zur Entwicklungszeit geschriebene Code ist nicht der Selbe, welcher zur Laufzeit im Browser ausgeführt wird, da dieser transpiliert und womöglich minifiziert wurde.
Wenn nun Fehler der Applikation zur Laufzeit identifiziert werden, können diese nicht mehr auf den Ursprungscode abgebildet werden.
Der Typescript Compiler beinhaltet einen SourceMaps Generator, welcher beim Transpilevorgang .map Dateien erzeugt,
welche dabei als Referenztabelle zwischen Quell und Zielcode fungieren.
Öffnet man nun die Entwicklerkonsole in einem Browser, welcher SourceMaps unerstützt, kann man den ursprünglichen Code inspizieren.
\cite{Using97:online}


\subsubsection{Komponenten verteilen}

Es wird eine Infrastruktur benötigt um Komponenten projektübergreifend verteilen zu können.
Möglich wäre die Konfiguration von Symlinks des lokalen Dateisystems um Komponenten in das Angular,
sowie Ionic Projekt zu integrieren. Symlinks lassen sich allerdings nicht versionieren, daher würde
für jeden Entwickler, der an dem Projekt mitwirken möchte,
ein hoher initialer Konfigurationsaufwand entstehen.
Des Weiteren sollen Komponenten womöglich nicht nur in der Ionic App,
sondern in vielen weiteren auf Angular 2 basierenden Projekten wiederverwendet werden.
Eine Komponente oder ein Paket diverser Komponenten soll veröffentlicht und aktuell
gehalten werden und soll von Entwicklern genutzt werden können.

Es liegt nahe dieses Problem mithilfe eines Paketmanagers zu lösen. Angular und Ionic verwenden für das Management
ihrer Kern-Abhängigkeiten den \ac{NPM}.
Open Source Pakete können damit kostenfrei veröffentlicht und aktualisiert werden. Closed Source Pakete können
bereits für einen Aufpreis von 7\$ pro Monat genutzt werden.


\subsubsection{Komponenten verwenden}






\paragraph{HTML und CSS Inlining}
Zudem werden \ac{HTML} und \ac{CSS} referenzen der Komponente in Inline-Strings konvertiert.
Damit werden Konflikte bezüglich relativen und absoluten Referenz-Pfaden verhindert,
da diese zentral in einer Angular Applikation konfiguriert werden.
Sind Style und Markup der Komponente als Inline-String definiert,
wird diese Konfiguration bezüglich der verteilten Komponenten hinfällig.
\cite{ludoh30:online}

\begin{figure}[h]
 \centering
 \includegraphics[width=\linewidth]{kapitel3/prepare-distribution-comp.png}
 \caption{Vergleich der Development und Distribution Komponente}
\end{figure}
\vspace{0.3cm}
