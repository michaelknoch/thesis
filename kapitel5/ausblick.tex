%!TEX root = ../thesis.tex

\chapter{Reflexion}
\label{chap:reflexion}




\section{Ausblick}

Während der Entwicklung des Systems entstanden fortlaufend neue projektrelevante Ideen,
die weit über die definierte Anforderungsanalyse des Prototypen hinausgingen.
Aufgrund des kurzen Entwicklungszeitraums konnten viele davon werder konzipiert noch implementiert werden.
Dennoch sollen diese im folgenden genannt und hinsichtlich ihrer Chancen kurz evaluiert werden.
Im folgenden wird zwischen funktionalen und nichtfunktionalen Ideen unterschieden.

\subsection{Funktional}

\subsubsection{Momente speichern}

Während der Implementierung des \emph{Meta Pickers} kam die Idee auf, Momente zur späteren Analyse speichern zu können.
Ein Use Case bestünde darin, dass ein Anwendungsentwickler und Nutzer der Applikation \projectname{}
über die mobile App eine Push Notification auf sein Smartphone erhält, weil auf seinem System ein problematischer Sachverhalt
entstanden ist. Mithilfe der App kann er sich zunächst Übersicht verschaffen und den Zeitpunkt des Problemauftritts speichern.
Im Büro wird der gespeicherte Zeitraum dann mithilfe der Desktop Anwendung vollständig analysiert,
das Problem wird von besagtem Anwendungsentwickler im Sourcecode ausfindig gemacht, behoben und der gespeicherte Zeitpunkt wird mit ``gelöst'' Markiert.

\subsubsection{Community}

Durch die Implementierung eines Community Features könnten System mithilfe von \projectname{} im Kollektiv überwacht werden.
Systeme müssten dahingehend für weitere Nutzer freigegeben werden können.
Zusätzlich könnten diverse Kommunikationsschnittstellen für einen
Austausch zwischen den Entwicklern implementiert werden. Ein möglicher Use Case könnte auf das Feature
\emph{Momente speichern} aufbauen,
indem Momente, beziehungsweise Zeiträume, nicht nur für die eigene Analyse gespeichert werden, sondern aus der mobilen Anwendung heraus
für eine Vielzahl von Entwicklern hinterlegt werden können.
Aufgrund von kritischem Systemverhalten entstehen Diskussionen um Probleme kollegial und damit effektiver lösen zu können.

\subsection{Nichtfunktional}


\subsubsection{Unit Testing}

Spätestens dann, wenn die Anwendung aktiv genutzt werden soll, darf ihre Glaubwürdigkeit nicht aufgrund von Fehlverhalten beeinflusst werden.
Unit Tests für Angular 2 Applikationen können beispielsweise mithilfe der Testing Bibliothek \emph{Jasmine} implementiert werden.
Dabei können einzelne Komponenten, Services oder zusätzliche Klassen vollständig getestet werden \cite{Angul78:online}.
Während der Entwicklung des Prototypen \projectname{} wurde die Entwicklung von Tests aufgrund besagter Zeitroblematik vernachlässigt,
daher sollte bestehende Funktionalität nachträglich, sowie neue Features durch Unit Tests abgedeckt werden um die Qualität der Software zu gewährleisten.

\subsubsection{Continuous Deployment}

Der Build Prozess, welcher momentan Lokal auf dem Computer des Entwicklers stattfindet, könnte auf einen externen Continuous Integration (CI) Server ausgelagert werden.
Denkbar wären Services wie \emph{TravisCI}, \emph{CircleCI} oder die Konfiguration eines eigenen Jenkins Server.
Änderungen in Form von Commits auf der Master Branch könnten damit automatisch die Test- und Buildkette starten um Updates für die Desktop Anwendung sowie für die App zu generieren.
Herausforderung dabei sind die Bedienung der verschiedenen Zielplattformen, da MacOS Desktop Anwendungen nur auf MacOS und
Windows Anwendungen ebenfalls nur auf Windows Plattformen generiert und signiert werden können.
Genannte CI-Services erlauben zwar eine Emulierung verschiedener Betriebsysteme,
allerdings bedeutet dies einen hohen Konfigurationsaufwand und erfordert ein gewisses Budget für die nötige Rechenleistung.
