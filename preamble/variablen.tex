\newcommand{\thema}{Design und Implementierung einer komponentenbasierten Cross-Plattform Frontend Architektur mit Angular2, Ionic2 und Electron in Typescript}
\newcommand{\schlagworte}{Cross-Plattform, Frontend Architektur, Angular2, Ionic2, Electron, Hybrid, Typescript}
\newcommand{\zusammenfassung}{Bei der Planung von Softwareprojekten ist die Frage der Zielplattform meist eine der ersten und bedeutendsten Fragen,
die gelöst werden muss. Eine Entscheidung darüber, welche Geräte und Betriebssysteme eine Applikation erreichen soll,
beeinflusst nicht nur nachhaltig die Zielgruppe der Anwender, sie definiert darüber hinaus wichtige Grundvoraussetzungen der Entwicklung.
Diese Bachelorarbeit befasst sich mit der Konzeption und Implementierung einer Frontend Architektur,
um mit möglichst geringem Programmieraufwand Endprodukte für Web, Desktop und Mobile zu erhalten.
Anhand der praktischen Arbeit werden Chancen und Risiken von Web-Technologien hinsichtlich Cross-Plattform Entwicklung aufgezeigt und evaluiert sowie mögliche Einschränkungen
erläutert. Web Components sind dabei von zentraler Bedeutung. Die Applikation soll komponentenbasiert entwickelt werden, um Elemente lückenlos für genannte Plattformen wieder
verwenden zu können. Speziell wird geprüft, in wie weit sich Funktionalität durch die Verwendung von Web Components und Typescript voneinander abstrahieren und bezüglich der Applikationsarchitektur
strukturieren lassen. Mithilfe der Frameworks Angular 2 und Ionic 2, welches ebenfalls auf Angular 2 aufbaut, sollen Anwendungen für Webbrowser sowie für iOS und Android entwickelt werden.
Zusätzlich soll mithilfe des Electron Framework eine Desktop Applikation für MacOS und Windows implementiert werden.}
\newcommand{\ausgabedatum}{15.05.2016}
\newcommand{\abgabedatum}{15.09.2016}
\newcommand{\autor}{Michael Knoch}
\newcommand{\autorStrasse}{Hornisgrindeweg 20}
\newcommand{\autorPLZ}{78050}
\newcommand{\autorOrt}{Villingen-Schwenningen}
\newcommand{\autorGeburtsort}{Villingen-Schwenningen}
\newcommand{\autorGeburtsdatum}{06.06.1992}
\newcommand{\prueferA}{Prof. Dr. Marko Boger}
\newcommand{\prueferB}{Prof. Dr. Markus Eiglsperger}
\newcommand{\firma}{HTWG}
\newcommand{\studiengang}{Software-Engineering}
\newcommand{\projectname}{MIA}
