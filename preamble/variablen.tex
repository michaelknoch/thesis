\newcommand{\thema}{Design und Implementierung einer komponentenbasierten Cross-Plattform Frontend Architektur mit Angular2, Ionic2 und Electron in Typescript}
\newcommand{\schlagworte}{Cross-Plattform, Frontend Architektur, Angular2, Ionic2, Electron, Hybrid, Typescript}
\newcommand{\zusammenfassung}{Als praktischer Teil der Arbeit soll eine Frontend Architektur für ein Monitoring System für Microservices konzipiert und implementiert werden, um
mit nahezu einer Codebasis Endprodukte für Web, Desktop und Mobile zu erhalten.
Anhand der praktischen Arbeit sollen Chancen und Risiken von Web Technologien hinsichtlich Cross Plattform Entwicklungen aufgezeigt und evaluiert sowie mögliche Einschränkungen
erläutert werden. Web Komponenten sind dabei von zentraler Bedeutung. Die Applikation soll komponentenbasiert entwickelt werden, um Elemente lückenlos für genannte Plattformen wieder
verwenden zu können. Speziell soll geprüft werden, wie weit sich Komponenten, durch die Verwendung von Typescript, voneinander abstrahieren und bezüglich der Applikationsarchitektur
strukturieren lassen. Mithilfe der Frameworks Angular 2 und Ionic, welches ebenfalls auf Angular 2 aufbaut, sollen Plattformen für Web sowie iOS und Android entwickelt werden.
Zusätzlich soll mittels Electron Framework eine Plattform für native Desktop Applikation für MacOS und Windows geschaffen werden.}
\newcommand{\ausgabedatum}{15.05.2016}
\newcommand{\abgabedatum}{15.09.2016}
\newcommand{\autor}{Michael Knoch}
\newcommand{\autorStrasse}{Bücklestr. 82}
\newcommand{\autorPLZ}{78467}
\newcommand{\autorOrt}{Konstanz}
\newcommand{\autorGeburtsort}{Villingen-Schwenningen}
\newcommand{\autorGeburtsdatum}{06.06.1992}
\newcommand{\prueferA}{Prof. Dr. Marko Boger}
\newcommand{\prueferB}{Prof. Dr. Markus Eiglsperger}
\newcommand{\firma}{HTWG}
\newcommand{\studiengang}{Software-Engineering}
\newcommand{\projectname}{MIA}
